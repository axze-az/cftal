\documentclass[10pt,a4paper,final,oneside]{article}
\usepackage{amsmath}
\usepackage[utf8]{inputenc}
\usepackage{color}
\usepackage[final]{listings}
\usepackage[bookmarks=true,hidelinks]{hyperref}
\usepackage[margin=1in]{geometry}
\usepackage{mathtools}
\numberwithin{equation}{subsection}


\lstset{
  % change this to 11 later
  language={C++},
  basicstyle=\footnotesize,
  keywordstyle=\bfseries\color{blue},
  identifierstyle=\color{black}
}

\begin{document}
\title{cftal - yet another short vector library}
\author{axel zeuner
\footnote{I am not a native english speaker, corrections are welcome, axel\ .\ zeuner\ @\ gmx\ .\ de}}
\maketitle
\setcounter{tocdepth}{4}
\setcounter{secnumdepth}{3}

\tableofcontents

\section{Introduction}
\label{sec:introduction}

The cftal library contains a C++11(+) short vector implementation. A
short vector is vector with up to 64 elements.

The design principles of the short vector library are
templates, recursion and specialization:
\begin{itemize}
\item A vector of length $N$ contains two vectors of length $N/2$
  (this also forces the library to vector lengths of powers of 2).
\item The vectors of length $1$ are specialized
\item Vectors with the same length as hardware vector implementation
  for different system may be specialized.
\item Partial and full specialization may be used to allow the
  selection of special intrinsic functions for hardware vector
  implementations
\item Expression templates are used for operations on the short
  vectors. This allows the filtering for special operation patterns,
  for instance generation of fused multiply and add intrinsics from
  expressions like $ c= a* b + c $. More optimizations are possible,
  but not (yet) implemented.
\end{itemize}

\section{Base components}
\label{sec:base}

All short vectory library code is grouped around the recursive template
class
\begin{lstlisting}
template <typename _T, std::size_t _N> vec<_T, _N>;
\end{lstlisting}
where $\_T$ is the type and $\_N$ the number of the elements of the
vector. The partial specialized template class
\begin{lstlisting}
template <typename _T> vec<_T, 1>;
\end{lstlisting}
is defined to stop the recursion.\\
%
A vector of length $N$ may be constructed from
\begin{itemize}
\item a scalar, initializing every vector element to the scalar,
\item two vectors of length $N/2$ for $N/2>=1$
      using a constructor like
\begin{lstlisting}
vec(const vec<_T, _N/2>& lh, const vec<_T, _N/2>& hh);
\end{lstlisting}
      where lh is the lower half and hh the higher half of the vector to
      construct,
\item an initializer list, initializing excess elements in the
  vector from the last element of the initializer list.
\end{itemize}
A vector may be split into low and high halfes using the functions
\begin{lstlisting}
template <typename _T, std::size_t _N>
const typename vec<_T, _N>::half_type&
low_half(const vec<_T, _N>& v);

template <typename _T, std::size_t _N>
const typename vec<_T, _N>::half_type&
high_half(const vec<_T, _N>& v);
\end{lstlisting}{\ }\\[10pt]
The most arithmetic operations are defined on all types of vectors.
Logical operations on vectors return either a bit mask or a vector of
the same type as the operands. The choice of a bit mask (a vector of
bits) or the same type depends on the machine -- machines with
hardware bit mask should use a bit mask (this means, that on x86 the
type of the result of logical operations depends on the existence of
the 512 bit AVX vector extensions). \\[10pt]
%
To suppress the generation of fused multiply and add instruction
in expressions like
\begin{lstlisting}
vec<double, 2> r0= a * b + c;
vec<double, 2> r1= a * b - c;
vec<double, 2> r2= c - a * b;
\end{lstlisting}
the result of the multiplication should be assigned to a (temporary)
variable
\begin{lstlisting}
vec<double, 2> t=a*b;
vec<double, 2> r1= t + c;
vec<double, 2> r2= t - c;
vec<double, 2> r3= c - t ;
\end{lstlisting}
which breaks the expression tree and avoids the generation of these
instructions (if the compiler is not too clever).\\[10pt]
%
Do not use vector lengths greater than 1 and smaller than the smallest
specialized vector length on your target machine, for instance avoid on
x86 machines vector lengths of 2 for float, int32\_t and uint32\_t and
vector lengths of 4 for int16\_t and uint16\_t
(these are cases where the compilers are not clever enough).\\[10pt]
%
A large number of functions exist only for vector classes of the
two floating point types double (binary64) and float (binary32).\\
The file \texttt{include/cftal/vec\_f64.h} declares the functions
for vectors of binary64 and also contains inline implementation of
the most functions, the file \texttt{include/cftal/vec\_f32.h} contains
the corresponding float functions and their implementations.

\section{Vector Math Library}
\label{sec:vec_math_lib}

This section contains the description of the algorithms used in the
vector math library. The implemented algorithms are the same or
heavily based on the algorithms used in the sun math
library. Therefore the copyright of the sun libm source codes
\small
\begin{verbatim}
/*
 * ====================================================
 * Copyright (C) 2004 by Sun Microsystems, Inc. All rights reserved.
 *
 * Permission to use, copy, modify, and distribute this
 * software is freely granted, provided that this notice
 * is preserved.
 * ====================================================
 */
\end{verbatim}
\normalsize
applies.\\[10pt]
%
Functions for vector lengths from 1 to 8 for binary64 and
1 to 16 for binary32 are implemented as direct instantiations of
the template code in
\lstinline{cftal/math/elem_func.h}
and in \lstinline{cftal/math/spec_func.h}
using the macros in
\lstinline{src/vec_def_math_functions.h}.
The binary32 functions are additionally instantiated for vector length 16.
The implementation of the instantiated functions is located in
\lstinline{cftal/math/elem_func_core_f64.h} and
\lstinline{cftal/math/elem_func_core_f32.h}.
%%
Vector functions with greater length are constructed using the template
\begin{lstlisting}
template <typename _T, std::size_t _N>
cftal::vec<_T, _N>
cftal::func(const vec<_T, _N>& x)
{
    vec<_T, _N> r= vec<_T, _N>(func(low_half(r)),
                               func(high_half(x)));
    return r;
}
\end{lstlisting}
recursively, where \textbf{func} is the name of the
function to construct.\\[10pt]
The polynomial approximations used in the implemented functions
are produced using sollya \cite{ChevillardJoldesLauter2010}.

\subsection{List of functions and precisions}
\label{sub_sec:func_list}
The tables in this section contain the list of implemented functions and
their measured maximium deviations against the corresponding functions
from the mpfr library. The column about faithfully rounding describes if faithfully rounding was observed using a test data set.
All test programs use the routines from The GNU MPFR library, see
\cite{mpfr}, \cite{Fousse:2007:MMB:1236463.1236468}, as reference
implementation.\\[10pt]
MPFR functions return a function value $res$ rounded to target precision
and information about the direction of the final rounding $mpfr\_res$.
The formulaes
\[
    \begin{aligned}
        \left[\text{std::nextafter}(res, -\infty), res\right] &
            \, \text{ when } mpfr\_res > 0 \\
        \left[res, \text{std::nextafter}(res, +\infty)\right] &
            \, \text{ when } mpfr\_res < 0 \\
        \left[res, res\right] &
            \, \text{ when } mpfr\_res = 0
    \end{aligned}
\]
describe the conversion between the values $res$ and $mpfr\_res$
and an interval.\\[10pt]
The column $\Delta$ulp in the subsections below shows the maximum
deviation in bits against the GNU MPFR library functions observed,
this means functions with $\Delta$ulp = 0 are correctly rounded.
Faithful rounding is achieved if the implemented function returns
one of the 2 interval borders.
%
%Functions with names like native\_xxx implement the function xxx with
%reduced precision or range or both.
%
\subsubsection{support functions}
\begin{tabular}{ | p{2.0cm} | p{2.0cm} | p{2.0cm} | p{2.0cm} | p{2.0cm} |}
    \hline
     &
    \multicolumn{2}{| p{4.0cm} |} {\center{binary64}} &
    \multicolumn{2}{| p{4.0cm} |} {\center{binary32}} \\
    \hline
    function & $\Delta$ulp & (prob.) faithfully &
          $\Delta$ulp & (prob.) faithfully \\
    \hline
    ldexp & $\pm$ 0 & n/a  & $\pm$ 0 & n/a \\
    \hline
    frexp & $\pm$ 0 & n/a  & $\pm$ 0 & n/a \\
    \hline
    ilogb & $\pm$ 0 & n/a  & $\pm$ 0 & n/a \\
    \hline
    isnan & $\pm$ 0 & n/a  & $\pm$ 0 & n/a \\
    \hline
    isinf & $\pm$ 0 & n/a  & $\pm$ 0 & n/a \\
    \hline
    isfinite & $\pm$ 0 & n/a  & $\pm$ 0 & n/a \\
    \hline
    copysign & $\pm$ 0 & n/a  & $\pm$ 0 & n/a \\
    \hline
    abs/fabs & $\pm$ 0 & n/a  & $\pm$ 0 & n/a \\
    \hline
    floor & $\pm$ 0 & n/a  & $\pm$ 0 & n/a \\
    \hline
    ceil & $\pm$ 0 & n/a  & $\pm$ 0 & n/a \\
    \hline
    trunc & $\pm$ 0 & n/a  & $\pm$ 0 & n/a \\
    \hline
    rint & $\pm$ 0 & n/a  & $\pm$ 0 & n/a \\
    \hline
    fmin & $\pm$ 0 & n/a  & $\pm$ 0 & n/a \\
    \hline
    fmax & $\pm$ 0 & n/a  & $\pm$ 0 & n/a \\
    \hline
    nextafter & $\pm$ 0 & n/a  & $\pm$ 0 & n/a \\
    \hline
\end{tabular}\\[10pt]
All these functions behave as their counterparts in the std namespace.

\subsubsection{power functions}
\begin{tabular}{ | p{2.0cm} | p{2.0cm} | p{2.0cm} | p{2.0cm} | p{2.0cm} |}
    \hline
     &
    \multicolumn{2}{| p{4.0cm} |} {\center{binary64}} &
    \multicolumn{2}{| p{4.0cm} |} {\center{binary32}} \\
    \hline
    function & $\Delta$ulp & (prob.) faithfully &
          $\Delta$ulp & (prob.) faithfully \\
    \hline
    rsqrt & $\pm$ 1 & y  & $\pm$ 1 & y \\
    \hline
    cbrt & $\pm$ 1 & y  & $\pm$ 1 & y \\
    \hline
    rcbrt & $\pm$ 1 & y  & $\pm$ 1 & y \\
    \hline
    hypot & $\pm$ 1 & y  & $\pm$ 1 & y \\
    \hline
    rootn & $\pm$ 1 & y  & $\pm$ 1 & y \\
    \hline
    powi & $\pm$ 1 & y  & $\pm$ 1 & y \\
    \hline
    pow & $\pm$ 1 & y  & $\pm$ 1 & y \\
    \hline
\end{tabular}\\[10pt]
The sqrt function uses the hardware instruction provided by modern processors.
The function $ rsqrt(x) $ calculates $ \frac{1}{\sqrt{x}}$, the function
$ rcbrt(x) $ $ \frac{1}{\sqrt[3]{x}}$.
The function $ rootn(x, n) $ calculates $ \sqrt[n]{x} $ and
the function $ powi(x, n) $ calculates $ x^n $, where $n$ is an integer.
%The precision of the pow function in more common areas, for instance in
%$(0, 100), (0, 100)$ is better than mentioned in the table.

\subsubsection{elementary functions}
\begin{tabular}{ | p{2.0cm} | p{2.0cm} | p{2.0cm} | p{2.0cm} | p{2.0cm} |}
    \hline
     &
    \multicolumn{2}{| p{4.0cm} |} {\center{binary64}} &
    \multicolumn{2}{| p{4.0cm} |} {\center{binary32}} \\
    \hline
    function & $\Delta$ulp & (prob.) faithfully &
          $\Delta$ulp & (prob.) faithfully \\
    \hline
    exp & $\pm$ 1 & y  & $\pm$ 1 & y \\
    \hline
    expm1 & $\pm$ 1 & y  & $\pm$ 1 & y \\
    \hline
    exp2 & $\pm$ 1 & y  & $\pm$ 1 & y \\
    \hline
    exp2m1 & $\pm$ 1 & y  & $\pm$ 1 & y \\
    \hline
    exp10 & $\pm$ 1 & y  & $\pm$ 1 & y \\
    \hline
    exp10m1 & $\pm$ 1 & y  & $\pm$ 1 & y \\
    \hline
    sinh & $\pm$ 1 & y  & $\pm$ 1 & y \\
    \hline
    cosh & $\pm$ 1 & y  & $\pm$ 1 & y \\
    \hline
    tanh & $\pm$ 1 & y  & $\pm$ 1 & y \\
    \hline \hline
    log & $\pm$ 1 & y  & $\pm$ 1 & y \\
    \hline
    log1p & $\pm$ 1 & y  & $\pm$ 1 & y \\
    \hline
    log2 & $\pm$ 1 & y  & $\pm$ 1 & y \\
    \hline
    log2p1 & $\pm$ 1 & y  & $\pm$ 1 & y \\
    \hline
    log10 & $\pm$ 1 & y  & $\pm$ 1 & y \\
    \hline
    log10p1 & $\pm$ 1 & y  & $\pm$ 1 & y \\
    \hline \hline
    sin & $\pm$ 1 & y  & $\pm$ 1 & y \\
    \hline
    cos & $\pm$ 1 & y  & $\pm$ 1 & y \\
    \hline
    tan & $\pm$ 1 & y  & $\pm$ 1 & y \\
    \hline \hline
    asin & $\pm$ 1 & y  & $\pm$ 1 & y \\
    \hline
    acos & $\pm$ 1 & y  & $\pm$ 1 & y \\
    \hline
    atan & $\pm$ 1 & y  & $\pm$ 1 & y \\
    \hline
    atan2 & $\pm$ 1 & y  & $\pm$ 1 & y \\
    \hline \hline
    asinh & $\pm$ 1 & y  & $\pm$ 1 & y \\
    \hline
    acosh & $\pm$ 1 & y  & $\pm$ 1 & y \\
    \hline
    atanh & $\pm$ 1 & y  & $\pm$ 1 & y \\
    \hline
\end{tabular}\\[10pt]
The functions exp2m1 and exp10m1 calculate $2^x-1$ and $10^x-1$ respectively,
the functions log2p1 and log10p1 calculate $log_2(1+x)$ and $log_{10}(1+x)$
respectively.

\subsubsection{additional elementary functions}
\begin{tabular}{ | p{2.0cm} | p{2.0cm} | p{2.0cm} | p{2.0cm} | p{2.0cm} |}
    \hline
     &
    \multicolumn{2}{| p{4.0cm} |} {\center{binary64}} &
    \multicolumn{2}{| p{4.0cm} |} {\center{binary32}} \\
    \hline
    function & $\Delta$ulp & (prob.) faithfully &
          $\Delta$ulp & (prob.) faithfully \\
    \hline
    exp\_mx2 & $\pm$ 1 & y  & $\pm$ 1 & y \\
    \hline
    exp\_px2 & $\pm$ 1 & y  & $\pm$ 1 & y \\
    \hline
    exp2\_mx2 & $\pm$ 1 & y  & $\pm$ 1 & y \\
    \hline
    exp2\_px2 & $\pm$ 1 & y  & $\pm$ 1 & y \\
    \hline
    exp10\_mx2 & $\pm$ 1 & y  & $\pm$ 1 & y \\
    \hline
    exp10\_px2 & $\pm$ 1 & y  & $\pm$ 1 & y \\
    \hline
    sig & $\pm$ 1 & y  & $\pm$ 1 & y \\
    \hline
    root12 & $\pm$ 1 & y  & $\pm$ 1 & y \\
    \hline
    sincos & $\pm$ 1 & y  & $\pm$ 1 & y \\
    \hline
    sinpi & $\pm$ 1 & y  & $\pm$ 1 & y \\
    \hline
    cospi & $\pm$ 1 & y  & $\pm$ 1 & y \\
    \hline
    tanpi & $\pm$ 1 & y  & $\pm$ 1 & y \\
    \hline
    sinpicospi & $\pm$ 1 & y  & $\pm$ 1 & y \\
    \hline
\end{tabular}\\[10pt]
The function exp\_mx2 calculates $ e^{-x^2}$, exp\_px2 calculates
$e^{x^2}$, the functions exp2\_mx2, exp2\_mx2, exp10\_mx2 and
exp10\_mx2 calculate $ 2^{-x^2}$, $ 2^{x^2}$, $ 10^{-x^2}$ and $
10^{x^2}$.
The logistic or sigmoid
function sig returns $\frac{1}{1+e^{-x}}$.
The functions sinpi, cospi and tanpi calculate $\sin(\pi *
x) $, $\cos(\pi * x) $ and $\tan(\pi * x) $ respectivly.  The function
sincos returns sinus and cosinus of its argument, sinpicospi returns
sinpi and cospi.

\subsubsection{special functions}
\begin{tabular}{ | p{2.0cm} | p{2.0cm} | p{2.0cm} | p{2.0cm} | p{2.0cm} |}
    \hline
     &
    \multicolumn{2}{| p{4.0cm} |} {\center{binary64}} &
    \multicolumn{2}{| p{4.0cm} |} {\center{binary32}} \\
    \hline
    function & $\Delta$ulp & (prob.) faithfully &
          $\Delta$ulp & (prob.) faithfully \\
    \hline
    erf & $\pm$ 1 & y  & $\pm$ 1 & y \\
    \hline
    erfc & $\pm$ 1 & y  & $\pm$ 1 & y \\
    \hline
    tgamma & $\pm$ 1 & y  & $\pm$ 1 & y \\
    \hline
    lgamma & $\pm$ 100? & n  & $\pm$ 100? & n \\
    \hline
    j0 & $\pm$ 1? & y?  & $\pm$ 1? & y? \\
    \hline
    j1 & $\pm$ 1? & y?  & $\pm$ 1? & y? \\
    \hline
    y0 & $\pm$ 1? & y?  & $\pm$ 1? & y? \\
    \hline
    y1 & $\pm$ 1? & y?  & $\pm$ 1? & y? \\
    \hline
\end{tabular}\\[10pt]
The function lgamma is probably faithfully rounded for non negative arguments.

\if 0

\subsection{Algorithms}
In the subsections below the notation $ lg(x) = \log_{10}(x)$,
$ ld(x) = \log_2{(x)} $ and $ \log{(x)} = \log_e{(x)} $ is used. A double pair
or double double variable is a floating point expansion of the form
\[
    (x_h, x_l) = x_h + x_l
\]
with $ |x_l| < \frac{1}{2} |x_h^{-p}| $, where p is the number of mantissa
bits in the floating variable.
\subsubsection{exponential functions}
\label{sub_sec:expxxx}

The exponential functions use additive argument reduction.
They use common code for the calculation of the exponential function
after reduction and scaling of the reduced argument for bases not
equal to $e$.

\paragraph{argument reduction for exp(x) and expm1(x)}

The input argument is split into
\begin{equation}
  x = k \times \log{(2)} + r_h + r_l, \;
  |r_h +r_l| \le \frac{1}{2} \log{(2)}
\end{equation}
whith $r = r_h + r_l$.
After the determination of
\[
k = x \times \frac{1}{\log{(2)}}
\]
the values $r_h$ and $r_l$ can be calculated as
\[
\begin{aligned}
  hi &= x - k \times LN2_{HI} \\
  r_h &= hi - k \times LN2_{LO} \\
  dx & = hi - r_h \\
  r_l &= dx - k \times LN2_{LO}
\end{aligned}
\]
where $LN2_{HI}$ and $LN2_{LO}$ are Cody and Waite constants for $\log(2)$
and the expressions are organized to allow utilization of fma operations.

\paragraph{argument reduction for exp2(x) and exp2m1(x)}

The calculations of exp2(x) and exp2m1(x) use the
identity
\begin{equation}
  2^x + C = e^{x \times \log{(2)}} + C
\end{equation}
for the reduced arguments.
The reduced argument $ r_0 $ satisfies
\begin{equation}
  x = k + r_0, \;
  |r_0| \le \frac{1}{2}
\end{equation}
After the determination of
\[
k = \text{rint}(x)
\]
the value $r_0$ is calculated as
\[
r_0 = x - k
\]
and then scaled
\[
(r_h, r_l) = r_0 - k \times (LN2_{HI}, LN2_{LO})
\]
to allow the calculation of $ 2^x + C $ as $ 2^k \times e^{r_h+ rl} + C$.
The constants $LN2_{HI}$ and $LN2_{LO}$ are the two parts of a double double number for $\log{(2)}$.

\paragraph{argument reduction for exp10(x) and expm10m1(x)}

The calculations of exp10(x) and exp10m1(x) use the
identity
\begin{equation}
  10^x + C = e^{x \times \log{(10)}} + C
\end{equation}
for the reduced arguments. A  double double pair $ r_0= r_{0h} + r_{0l} $ is constructed which satisfies
\begin{equation}
  x = k \times \log_{10}{(2)} + r_{0h} + r_{0l}, \;
  |xr_h +xr_l| \le \frac{1}{2} \log_{10}{(2)}
\end{equation}
After the determination of
\[
k = x \times \frac{1}{\log_{10}{(2)}}
\]
the values $r_{0h}$ and $r_{0l}$ can be calculated as
\[
\begin{aligned}
  hi &= x - k \times LG2_{HI} \\
  r_{0h} &= hi - k \times LG2_{LO} \\
  dx & = hi - r_{0h} \\
  r_{0l} &= dx - k \times LG2_{LO}
\end{aligned}
\]
where $LG2_{HI}$ and $LG2_{LO}$ are Cody and Waite constants for
$\log_{10}{(2)}$
and the expressions are organized to allow utilization of fma operations.
The reduced argument is then scaled
\[
\begin{aligned}
  (r_h, r_l) &= r_{0h} - k \times (LN10_{HI}, LN10_{LO}) \\
  r_l &= r_l + r_{0l} \times LN10_{HI}
\end{aligned}
\]
to allow the calculation of $ 10^x + C $ as $ 2^k * e^{r_h+ rl} + C$.
The constants $LN10_{HI}$ and $LN10_{LO}$ are the two parts of a
double double number for $\log(10)$.


\paragraph{approximation of $e^r$ in [$-\log(2)/2, \log(2)/2$]}

The function $y = e^r$ is approximated as
\begin{equation}
  y = \sum_{i=0}^{N}c_i r_h^i
\end{equation}
with $N=13$ for binary64 operands and $N=7$ for binary32 operands.
The summations with index 0 and 1 are done in higher precision
using an error free transformation of the horner's scheme
(also known as Compensated Horner Scheme), see \cite{Graillat05compensatedhorner}) to achieve
faithful rounded results.
%
Using the identity for the exponential function
\[
\begin{aligned}
  e^{r_h+r_l} - 1 &=
  e^{r_h} \times e^{r_l} - 1 \\
  &=
  e^{r_h} \times \big( 1+ r_l + \frac{r_l^2}{2} + \dots \big) - 1 \\
  &=
  e^{r_h} -1 + \big( e^{r_h} \times r_l +
  e^{r_h} \times \frac{r_l^2}{2} +
  \dots
  \big) \\
  &=
  e^{r_h} -1 + \big( r_l + (e^{r_h}-1) \times r_l +
  e^{r_h} \times \frac{r_l^2}{2} +
  \dots
  \big) \\
\end{aligned}
\]
and supressing all terms with higher powers of $r_l$ as 1 one yields
\begin{equation}
  \begin{aligned}
    e^{r_h+r_l} - 1 & \approx
    (e^{r_h} -1) + (r_l + (e^{r_h}-1) \times r_l)
  \end{aligned}
\end{equation}
as error correction formula.

\paragraph{reconstruction}

The reconstruction of all functions like $b^x$ is done as
\begin{equation}
  \begin{aligned}
    e ^ x & = & 2^k * e^{(r_h + r_l)} \\
          & = & 2^{k/2 + (k - k/2)} * e^{(r_h + r_l)} \\
          & = & 2^{k/2} * 2^{(k - k/2)} * e^{(r_h + r_l)}
  \end{aligned}
\end{equation}
The split of $k$ into $k/2$ and $k-k/2$ avoids the handling of
border cases for the maximum exponent and subnormal results.\\[10pt]
%
The reconstruction phase of all functions similiar to $b^x-1$ uses
with $ s  = \frac{1}{2} k $ the formulaes
\begin{equation}
  \begin{aligned}
    (e^{\frac{x_h}{2}}-\frac{1}{2}, e^{\frac{x_l}{2}}-\frac{1}{2})
    &=
    2^s \times (e^{r_h},e^{r_l}) - \frac{1}{2}\\
    &=
    (0.5 \times 2^k) \times (e^{r_h},e^{r_l}) - \frac{1}{2}
  \end{aligned}
\end{equation}
calculated in multiple precision. The two parts of
    $ e^{\frac{x}{2}} - \frac{1}{2} $ are then combined
\begin{equation}
  \begin{aligned}
    t &= 2 \times ( e^{\frac{x_h}{2}} -\frac{1}{2} ) \\
    e ^ x -1 &= t + 2 \times (e^{\frac{x_l}{2}} -\frac{1}{2})
  \end{aligned}
\end{equation}
into the final result. This rather complicated approach avoids the
handling of different cases for k, especially overflows during the
calculation.

\subsubsection{hyperbolic functions}

%The hyperbolic function sinh(x), cosh(x) and tanh(x) are approximated as
%described by Manos et al. in their report \cite{manos1972constrained}.
The hyperbolic functions $\sinh(x)$, $\cosh(x)$ and $\tanh(x)$
are approximated similiar to the quick phase of of the algorithm
described in crlibm \cite{crlibmweb}.
The implemented hyperbolic functions are defined by
\[
    \begin{aligned}
        \sinh(x) &= \frac{e^x -  e^{-x}}{2} \\
        \cosh(x) &= \frac{e^x +  e^{-x}}{2}
    \end{aligned}
\]
and
\[
    \tanh(x) = \frac{e^x -  e^{-x}}{e^x +  e^{-x}}
\]
respectively. Because of symmetries only positive values of $x$ are used
in the calculations.
The addition formulaes
\[
    \begin{aligned}
        \cosh(x + y) &= \cosh(x)  \cosh(y) + \sinh(x)  \sinh(y) \\
        \sinh(x + y) &= \sinh(x)  \cosh(y) + \sinh(y)  \cosh(x)
    \end{aligned}
\]
together with
\[
    \begin{aligned}
        \cosh(k\times\log(2)) &= 2^{k-1} + 2^{-k-1} \\
        \sinh(k\times\log(2)) &= 2^{k-1} - 2^{-k-1}
    \end{aligned}
\]
allow the argument reduction of $|x|$ as
$ |x|= k \times \log(2) + x_{rh} + x_{rl}$ with $x_r =x_{rh} + x_{rl}$.
This is done as in the case of the exponential function.
For the reduced argument $x_{rh}$ $\cosh(x_{rh})$ and $\sinh(x_{rh})$ are approximated as polynomials of order 12 and 13 for binary64 and of
order 7 and 6 for binary32. The last two steps of the polynomial
evaluation are done in double double precision.
The errors in the argument reduction are corrected using the first
order taylor expansions
\[
    \begin{aligned}
        \cosh(x_{rh} + x_{rl}) &\approx \cosh(x_{rh}) + \sinh(x_{rh}) x_{rl} \\
        \sinh(x_{rh} + x_{rl}) &\approx \sinh(x_{rh}) + \cosh(x_{rh}) x_{rl} \\
    \end{aligned}
\]
The eight products
\begin{itemize}
\item $2^{k-1} \times \cosh(x_r)_h$,
\item $2^{k-1} \times \cosh(x_r)_l$,
\item $2^{k-1} \times \sinh(x_r)_h$,
\item $2^{k-1} \times \sinh(x_r)_l$,
\item $2^{-k-1} \times \cosh(x_r)_h$,
\item $2^{-k-1} \times \cosh(x_r)_l$,
\item $2^{-k-1} \times \sinh(x_r)_h$ and
\item $2^{-k-1} \sinh(x_r)_l$
\end{itemize}
are exact.
The summation of these products is done in
double double precision.
The terms of order $ 2^{-k-1} $ are only calculated for values of
$ k <= 35 $ for binary64 and $ k <= 13 $ for binary32.
The case $k$ equal to the maximim possible exponent is handled
via  $k' = k - 1$ and a final multiplcation by 2. \\[10pt]
The function $\tanh(x)$ is calculated as quotient the obtained double double
precision values for $\sinh(x)$ and $\cosh(x)$ for $ k<=35 $ in binary64 and
$k <=13$ for binary32, otherwise $\pm 1$ is returned.


%
%TODO
%
\subsubsection{power functions}

The cbrt function ($ y=\sqrt[3]{x} $) was implemented as discussed by
W. Kahan in \cite{Kahan1991}.\\[10pt]
%
Inverse square root, to calculate $ y = 1/\sqrt{x} $ one may use the
iteration
\begin{equation}
    \begin{aligned}
        y &= y_n \left( 1+ ( x y_n^2 -1) \right)^{-1/2}
            = y_n \sum_{k=0} ({{-1/2}\atop{k}}) (x y_n^2 -1)^k \\
            & = y_n \left( 1 + \frac{1}{2} (1 - x y_n^2 ) +
            \frac{3}{8}(1- x y_n^2)^2 +
            \frac{5}{16} (1-x y_n^2)^3 \dots
            \right)
    \end{aligned}
\end{equation}
with a higher convergence rate than the newton raphson method.\\[10pt]
\subsubsection{logarithm functions}
%
The logarithm function is calculated using
\begin{equation}
    \begin{aligned}
        \log(x) &= \log(2^k \times x_r)  \\
                &= \log(2^k) + \log(x_r)  \\
                &= k \times \log(2) + \log(x_r)
  \end{aligned}
\end{equation}
and the right choice of k enforces
$ \frac{\sqrt{2}}{2} \le x_r \le \sqrt{2} $.
This may be done using standard function \mbox{frexp(x, \&k)} and doubling
the resulting mantissa and decrementing k by one if the mantissa is
smaller then $\frac{\sqrt{2}}{2}$.\\[10pt]
%
The logarithm of the reduced value $x_r$ is then calculated using the
identity
\begin{equation}
  \log(x) = \log \left( \frac{1+s}{1-s} \right) = \log(1+s) - \log(1-s)
\end{equation}
with $  s= \frac{x-1}{x+1} $.
The taylor series of the right side of the equation above:
\begin{displaymath}
\log(x) = 2s
\left( 1 + \frac{1}{3} s^2 + \frac{1}{5} s^4 + \frac{1}{7} s^6 + \dots  \right)
\end{displaymath}
is approximated by a polynomial of order 11 for binary32 and 19 for binary64.
%
The functions log2(x) and log10(x) are calculated as given by their definition
\[
    \begin{aligned}
        \log_{10}(x) &= \frac{\log(x)}{\log(10)}\\
        \log_{2}(x) &= \frac{\log(x)}{\log(2)}
    \end{aligned}
\]
from log(x) with higher precision with special handling for the integer part.

\subsection{trigonometric functions}

Let S,C and T denote the sin, cos and tan respectively on
$ \left[ -\pi/4, +\pi/4 \right] $.
Reduce the argument $x$ to $y1+y2 = x-k*\pi/2$
in $ \left[ -\pi/4, +\pi/4 \right] $, and let $n = k \mod 4 $.
The table\\[12pt]
\begin{tabular}{ | p{2.0cm} | p{2.5cm} | p{2.5cm} | p{2.5cm} |}
    \hline
    n & $\sin(x)$ &  $\cos(x)$ & $\tan(x)$ \\
    \hline
    0 & S &  C & T \\
    1 & C &  -S & -1/T \\
    2 & -S &  -C & T \\
    3 & -C &  S & -1/T \\
    \hline
\end{tabular}\\[12pt]
displays the values of the trigonometric functions for different $n$.

\subsection{All other functions}

    All other elementary functions use the algorithms from the sun
    math library. The polynomial approximations are recalculated
    using sollya.

\subsection{TODO}
    $[0,\,\frac{1}{2}log(2)]$:

    as

    \begin{equation}
        \begin{aligned}
            e^{r} &= 1 + r + \frac{r^2}{2} + \frac{r^3}{6} + \dots \\
                  &\approx  1 + \frac {2r} {2 -r + r^2 P(r^2)} \\
        \end{aligned}
    \end{equation}

    To obtain higher precision the last equation is rewritten as
    \begin{equation}
        e^{r} \approx  1 + r + \frac{r [r -r^2 P(r^2)]} {2-[r-r^2P(r^2)]}
    \end{equation}

    Furthermore a correction term
    \[
        c = (r_h - r) - r_l
    \]
    is calculated.

    Using the taylor series of $e^{r+c}-1$ for $c \ll r$ and and truncating
    higher order terms
    \begin{equation}
        \label{equ:expm1-taylor}
        \begin{aligned}
            e^{r+c}-1 &= (e^r-1) + e^r\,c  + \frac{e^r c^2}{2} + \dots \\
                      &= (e^r-1) + e^r\,c  + \dots \\
                      &= (e^r-1) + (1+r+\frac{r^2}{2}+\dots)\,c + \dots \\
                      &= (e^r-1) + (c + r\,c + \dots) + \dots \\
                      &\approx (e^r-1) + c + r\,c
        \end{aligned}
    \end{equation}
    $e^{r+c}$ is calculated as
    \begin{equation}
        \begin{aligned}
            s &= \frac{r [r -r^2 P(r^2)]} {2-[r-r^2P(r^2)]} \\
            s &= s + r*c \\
            s &= s + c \\
            s &= s + r \\
            e^{r+c} &= s + 1
        \end{aligned}
    \end{equation}


    The order of the minimax polynomial $P(r^2)$ is 10 for double precision.

 The function value of the reduced argument is scaled back using
    \begin{equation}
        e^x = 2^k \times e^r
    \end{equation}
    to obtain the result.



double argument identity for exponential functions:
\[
    \begin{aligned}
        (b^x-1)^2 &= b^{2x} - 2b^x + 1 \\
        b^{2x}    &= (b^x-1)^2 + 2b^x - 1 \\
        b^{2x} -1 &= (b^x-1)^2 + 2b^x -2 \\
                  &= (b^x-1)^2 + 2(b^x-1)
    \end{aligned}
\]

argument transformation from $ [a, b] $ into interval $ [-1, 1] $:
\[
    \tilde x = \frac{1}{2} \big[(b-a) x + a +b \big]
\]
\\[10pt]
mapping of the interval $ [0, \infty] $ to the interval $ [-1, 1] $ for
calculation of the error function
\[
    \tilde x = \frac{x-k}{x+k}
\]
with $ k = 3.75 $.


\subsection{expm1}
\label{sub_sec:expm1}
The function expm1(x) calculates $ e^x-1 $.

\subsubsection{precision}
The functions have an error of $ \pm 1\, ulp$.

\subsubsection{implementation}
\begin{itemize}
\item Argument reduction

    Reduce $x$ to $r$ so that $ |r| \le \frac{1}{2} log(2) $
    \begin{equation}
        x = k \times log(2) + r, \; |r| \le \frac{1}{2} log(2)
    \end{equation}
    $r$ is split into two values for calculation purposes
    \[
       \begin{aligned}
       r_h &= x - k \times LN2_{HI} \\
       r_l &= k \times LN2_{LO} \\
       r &= r_h - r_l
       \end{aligned}
    \]
    where $LN2_{HI}$ and $LN2_{LO}$ are Cody and Waite constants for $log(2)$.
    Furthermore a correction term
    \[
        c = (r_h - r) - r_l
    \]
    is calculated.

\item Approximation of $e^r-1$ by a special rational function on the interval
    $[0,\,\frac{1}{2}log(2)]$:
    \[
        r \frac{e^{r}+1}{e^{r}-1} =
        2+\frac{r^2}{6}-\frac{r^4}{360}+\cdots
    \]

\end{itemize}

\subsection{exp2}
The function exp2(x) calculates $ 2^x $.

\subsection{pow}
The function pow(x,y) calculates $ x^y $.

\subsubsection{precision}
The functions have an error of $ \pm x\, ulp$.

\subsubsection{implementation}
\begin{itemize}
\item argument reduction
    \begin{equation}
        \begin{aligned}
        x^y &= (2^{k_x} \times m_x) ^ {2^{k_y} \times m_y} \\
        log(x^y) &= y*log(x) \\
                 &= (2^{k_y} log(m_x) + log(2) k_x 2^{k_y}) m_y \\
                 &= (log(m_x) + log(2) k_x) 2^{k_y} m_y \\
        \end{aligned}
    \end{equation}
\end{itemize}

\subsection{cbrt}

\subsection{asinh}
\label{sub_sec:asin}
The function asin(x) calculates $ asinh(x) $.

\subsubsection{precision}
The functions have an error of $ \pm 2\, ulp$.

\subsubsection{implementation}
\begin{itemize}
\item Argument reduction
    \[
        \begin{aligned}
          asinh(x) &= log(x+\sqrt{1+x^2}) \\
                   &= log(x+\sqrt{1-x}\sqrt{1+x}) \\
          asinh(u) + asinh(v) &= asinh(u\sqrt{1+v^2} + v\sqrt{1+u^2}) \\
          asinh(x) + asinh(x) &= asinh(x\sqrt{1+x^2} + x\sqrt{1+x^2}) \\
                              &= asinh(2\,x\sqrt{1+x^2}) \\
          x &= 2^k\times r
        \end{aligned}
    \]
\end{itemize}

\fi

\bibliographystyle{plain}
\bibliography{cftal-doc}

\end{document}

